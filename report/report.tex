\documentclass[12pt]{article}

\usepackage[utf8]{inputenc}
\usepackage[russian]{babel}
\usepackage{tikz}
\usepackage{pgfplots}
\usepackage{longtable}
\usepackage{amsfonts} % for natural numbers, prime numbers sets
% \usepackage{seqsplit}

% No indentation for paragraphs
\setlength\parindent{0pt}

\usepackage{amsmath}
\usepackage{tabularx}
\usepackage[a4paper, total={6in, 9in}]{geometry}

\newcommand{\F}{\mathbb{F}_2}
\newcommand{\Fq}{\mathbb{F}_2^q}
\newcommand{\E}{\mathbb{E}}
\newcommand{\defeq}{\overset{\underset{\mathrm{def}}{}}{=}}


\newcommand{\SpeedPlot}[4]{
    \begin{center}
        \begin{tikzpicture}
            \begin{axis}[
                    xlabel = {количество исправляемых кодом ошибок t},
                    ylabel = {скорость БЧХ-кода $r=\frac{k}{n}$},
                    width = \linewidth,
                    height = 316,
                    % tick label style={font=\tiny},
                    grid = both,
                    #2
                ]
                \addplot[mark=#3,blue] plot coordinates {
                    #4
                };
                \addlegendentry{$r(t) = \frac{k(t)}{n}, \text{при } n = {#1}$}
            \end{axis}
        \end{tikzpicture}
    \end{center}
}

\newcommand{\TimePlot}[4]{
    \begin{center}
        \begin{tikzpicture}
            \begin{axis}[
                    xlabel = {количество исправляемых кодом ошибок t},
                    ylabel = {время передачи},
                    width = \linewidth,
                    height = 316,
                    % tick label style={font=\tiny},
                    grid = both,
                    #2
                ]
                \addplot[mark=#3,blue] plot coordinates {
                    #4
                };
                \addlegendentry{$\lambda = \frac{1}{rt}, \text{при } n = {#1}$}
            \end{axis}
        \end{tikzpicture}
    \end{center}
}


\title{\textbf{Курс: Прикладная алгебра, осень 2018 \\
Практическое задание. Конечные поля и коды БЧХ \\}}
\author{Работу выполнил\\
\textbf{Бобко Н. А.} \\
\textbf{студент 323 группы}}
\date{Москва, 2018}

\begin{document}

    \maketitle
    \newpage

    \tableofcontents
    \newpage

    \section{Формулировка задания}
        В задании выдаётся список всех примитивных многочленов степени $q$ над полем $\F$ для всех q = 2, ... , 16.
        В этом списке каждый многочлен представлен десятичным числом, двоичная запись которого соответствует
        коэффициентам полинома над $\F$, начиная со старшей степени.\\

        \textbf{Требуется:}\\
        1. Реализовать основные операции в поле $\Fq$: сложение, умножение, деление, решение СЛАУ, поиск 
        минимального многочлена из $\F[x]$ для заданного набора корней из поля $\Fq$; \\
        2. Реализовать основные операции для работы с многочленами из $\Fq[x]$: произведение многочленов, 
        деление многочленов с остатком, расширенный алгоритм Евклида для пары многочленов, вычисление 
        значения многочлена для набора элементов из $\Fq$; \\
        3. Реализовать процедуру систематического кодирования для циклического кода, заданного своим 
        порождающим многочленом; \\
        4. Реализовать процедуру построения порождающего многочлена для БЧХ-кода при заданных $n$ и 
        $t$; \\
        5. Построить графики зависимости скорости БЧХ-кода $r=\frac{k}{n}$ от количества исправляемых 
        кодом ошибок $t$ для различных значений $n$. Какие значения $t$ следует выбирать на практике
        для заданного $n$? \\
        6. Реализовать процедуру вычисления истинного минимального расстояния циклического кода $d$, заданного
        своим порождающим многочленом, путем полного перебора по всем $2^k - 1$ кодовым словам. Привести
        пример БЧХ-кода, для которого истинное минимальное расстояние больше, чем величина $2t + 1$;
        7. Реализовать процедуру декодирования БЧХ-кода с помощью метода PGZ и на основе расширенного 
        алгоритма Евклида. Провести сравнение двух методов декодирования по времени работы;

    \section{Результаты выполнения}
        \subsection{Скорость БЧХ кода и выбор оптимального кода для заданного n}
            Ниже приведены графики зависимости скорости БЧХ-кода $r=\frac{k}{n}$ от количества исправляемых 
            кодом ошибок $t$ для различных значений $n$, где $k$ - кол-во ``полезных'' бит в кодовом слове, 
            несущих само сообщение, а $n$ - длина кодового слова. А также помимо графиков таблица, по 
            которой эти графики были построены. \\

            \SpeedPlot{3}{xtick = data}{*}{
                (1, 0.3333333333333333)
            }

            \SpeedPlot{7}{xtick = data}{*}{
                (1, 0.5714285714285714)
                (2, 0.14285714285714285)
                (3, 0.14285714285714285)
            }

            \SpeedPlot{15}{xtick = data,}{*}{
                (1, 0.7333333333333333)
                (2, 0.4666666666666667)
                (3, 0.3333333333333333)
                (4, 0.06666666666666667)
                (5, 0.06666666666666667)
                (6, 0.06666666666666667)
                (7, 0.06666666666666667)
            }

            \SpeedPlot{31}{xtick = data,}{*}{
                (1, 0.8387096774193549)
                (2, 0.6774193548387096)
                (3, 0.5161290322580645)
                (4, 0.3548387096774194)
                (5, 0.3548387096774194)
                (6, 0.1935483870967742)
                (7, 0.1935483870967742)
                (8, 0.03225806451612903)
                (9, 0.03225806451612903)
                (10, 0.03225806451612903)
                (11, 0.03225806451612903)
                (12, 0.03225806451612903)
                (13, 0.03225806451612903)
                (14, 0.03225806451612903)
                (15, 0.03225806451612903)
            }

            \SpeedPlot{63}{}{*}{
                (1, 0.9047619047619048)
                (2, 0.8095238095238095)
                (3, 0.7142857142857143)
                (4, 0.6190476190476191)
                (5, 0.5714285714285714)
                (6, 0.47619047619047616)
                (7, 0.38095238095238093)
                (8, 0.2857142857142857)
                (9, 0.2857142857142857)
                (10, 0.2857142857142857)
                (11, 0.25396825396825395)
                (12, 0.15873015873015872)
                (13, 0.15873015873015872)
                (14, 0.1111111111111111)
                (15, 0.1111111111111111)
                (16, 0.015873015873015872)
                (17, 0.015873015873015872)
                (18, 0.015873015873015872)
                (19, 0.015873015873015872)
                (20, 0.015873015873015872)
                (21, 0.015873015873015872)
                (22, 0.015873015873015872)
                (23, 0.015873015873015872)
                (24, 0.015873015873015872)
                (25, 0.015873015873015872)
                (26, 0.015873015873015872)
                (27, 0.015873015873015872)
                (28, 0.015873015873015872)
                (29, 0.015873015873015872)
                (30, 0.015873015873015872)
                (31, 0.015873015873015872)
            }

            \SpeedPlot{127}{}{}{
                (1, 0.9448818897637795)
                (2, 0.889763779527559)
                (3, 0.8346456692913385)
                (4, 0.7795275590551181)
                (5, 0.7244094488188977)
                (6, 0.6692913385826772)
                (7, 0.6141732283464567)
                (8, 0.5590551181102362)
                (9, 0.5590551181102362)
                (10, 0.5039370078740157)
                (11, 0.44881889763779526)
                (12, 0.3937007874015748)
                (13, 0.3937007874015748)
                (14, 0.33858267716535434)
                (15, 0.28346456692913385)
                (16, 0.2283464566929134)
                (17, 0.2283464566929134)
                (18, 0.2283464566929134)
                (19, 0.2283464566929134)
                (20, 0.2283464566929134)
                (21, 0.2283464566929134)
                (22, 0.1732283464566929)
                (23, 0.1732283464566929)
                (24, 0.11811023622047244)
                (25, 0.11811023622047244)
                (26, 0.11811023622047244)
                (27, 0.11811023622047244)
                (28, 0.06299212598425197)
                (29, 0.06299212598425197)
                (30, 0.06299212598425197)
                (31, 0.06299212598425197)
                (32, 0.007874015748031496)
                (33, 0.007874015748031496)
                (34, 0.007874015748031496)
                (35, 0.007874015748031496)
                (36, 0.007874015748031496)
                (37, 0.007874015748031496)
                (38, 0.007874015748031496)
                (39, 0.007874015748031496)
                (40, 0.007874015748031496)
                (41, 0.007874015748031496)
                (42, 0.007874015748031496)
                (43, 0.007874015748031496)
                (44, 0.007874015748031496)
                (45, 0.007874015748031496)
                (46, 0.007874015748031496)
                (47, 0.007874015748031496)
                (48, 0.007874015748031496)
                (49, 0.007874015748031496)
                (50, 0.007874015748031496)
                (51, 0.007874015748031496)
                (52, 0.007874015748031496)
                (53, 0.007874015748031496)
                (54, 0.007874015748031496)
                (55, 0.007874015748031496)
                (56, 0.007874015748031496)
                (57, 0.007874015748031496)
                (58, 0.007874015748031496)
                (59, 0.007874015748031496)
                (60, 0.007874015748031496)
                (61, 0.007874015748031496)
                (62, 0.007874015748031496)
                (63, 0.007874015748031496)
            }

            \SpeedPlot{255}{}{}{
                (1, 0.9686274509803922)
                (2, 0.9372549019607843)
                (3, 0.9058823529411765)
                (4, 0.8745098039215686)
                (5, 0.8431372549019608)
                (6, 0.8117647058823529)
                (7, 0.7803921568627451)
                (8, 0.7490196078431373)
                (9, 0.7333333333333333)
                (10, 0.7019607843137254)
                (11, 0.6705882352941176)
                (12, 0.6392156862745098)
                (13, 0.6078431372549019)
                (14, 0.5764705882352941)
                (15, 0.5450980392156862)
                (16, 0.5137254901960784)
                (17, 0.5137254901960784)
                (18, 0.5137254901960784)
                (19, 0.4823529411764706)
                (20, 0.45098039215686275)
                (21, 0.45098039215686275)
                (22, 0.4196078431372549)
                (23, 0.38823529411764707)
                (24, 0.3568627450980392)
                (25, 0.3568627450980392)
                (26, 0.3411764705882353)
                (27, 0.30980392156862746)
                (28, 0.2784313725490196)
                (29, 0.2784313725490196)
                (30, 0.24705882352941178)
                (31, 0.21568627450980393)
                (32, 0.1843137254901961)
                (33, 0.1843137254901961)
                (34, 0.1843137254901961)
                (35, 0.1843137254901961)
                (36, 0.1843137254901961)
                (37, 0.1843137254901961)
                (38, 0.1843137254901961)
                (39, 0.1843137254901961)
                (40, 0.1843137254901961)
                (41, 0.1843137254901961)
                (42, 0.1843137254901961)
                (43, 0.17647058823529413)
                (44, 0.1450980392156863)
                (45, 0.1450980392156863)
                (46, 0.11372549019607843)
                (47, 0.11372549019607843)
                (48, 0.08235294117647059)
                (49, 0.08235294117647059)
                (50, 0.08235294117647059)
                (51, 0.08235294117647059)
                (52, 0.08235294117647059)
                (53, 0.08235294117647059)
                (54, 0.08235294117647059)
                (55, 0.08235294117647059)
                (56, 0.050980392156862744)
                (57, 0.050980392156862744)
                (58, 0.050980392156862744)
                (59, 0.050980392156862744)
                (60, 0.03529411764705882)
                (61, 0.03529411764705882)
                (62, 0.03529411764705882)
                (63, 0.03529411764705882)
                (64, 0.00392156862745098)
                (65, 0.00392156862745098)
                (66, 0.00392156862745098)
                (67, 0.00392156862745098)
                (68, 0.00392156862745098)
                (69, 0.00392156862745098)
                (70, 0.00392156862745098)
                (71, 0.00392156862745098)
                (72, 0.00392156862745098)
                (73, 0.00392156862745098)
                (74, 0.00392156862745098)
                (75, 0.00392156862745098)
                (76, 0.00392156862745098)
                (77, 0.00392156862745098)
                (78, 0.00392156862745098)
                (79, 0.00392156862745098)
                (80, 0.00392156862745098)
                (81, 0.00392156862745098)
                (82, 0.00392156862745098)
                (83, 0.00392156862745098)
                (84, 0.00392156862745098)
                (85, 0.00392156862745098)
                (86, 0.00392156862745098)
                (87, 0.00392156862745098)
                (88, 0.00392156862745098)
                (89, 0.00392156862745098)
                (90, 0.00392156862745098)
                (91, 0.00392156862745098)
                (92, 0.00392156862745098)
                (93, 0.00392156862745098)
                (94, 0.00392156862745098)
                (95, 0.00392156862745098)
                (96, 0.00392156862745098)
                (97, 0.00392156862745098)
                (98, 0.00392156862745098)
                (99, 0.00392156862745098)
                (100, 0.00392156862745098)
                (101, 0.00392156862745098)
                (102, 0.00392156862745098)
                (103, 0.00392156862745098)
                (104, 0.00392156862745098)
                (105, 0.00392156862745098)
                (106, 0.00392156862745098)
                (107, 0.00392156862745098)
                (108, 0.00392156862745098)
                (109, 0.00392156862745098)
                (110, 0.00392156862745098)
                (111, 0.00392156862745098)
                (112, 0.00392156862745098)
                (113, 0.00392156862745098)
                (114, 0.00392156862745098)
                (115, 0.00392156862745098)
                (116, 0.00392156862745098)
                (117, 0.00392156862745098)
                (118, 0.00392156862745098)
                (119, 0.00392156862745098)
                (120, 0.00392156862745098)
                (121, 0.00392156862745098)
                (122, 0.00392156862745098)
                (123, 0.00392156862745098)
                (124, 0.00392156862745098)
                (125, 0.00392156862745098)
                (126, 0.00392156862745098)
                (127, 0.00392156862745098)
            }

            \begin{center}
                \tiny
                \begin{longtable}{|c|c|c|c|c|c|c|c|}
                    \hline
                    \multicolumn{8}{|c|}{Зависимость скорости БЧХ-кода $r=\frac{k}{n}$ от количества} \\
                    \multicolumn{8}{|c|}{исправляемых кодом ошибок $t$ для различных значений $n$} \\
                    \hline
                    $t$ & $n = 3$ & $n = 7$ & $n = 15$ & $n = 31$ & $n = 63$ & $n = 127$ & $n = 255$ \\
                    \hline
                    1 & 0.3333 & 0.5714 & 0.7333 & 0.8387 & 0.9047 & 0.9448 & 0.9686 \\
                    2 &  & 0.1428 & 0.4666 & 0.6774 & 0.8095 & 0.8897 & 0.9372 \\
                    3 &  & 0.1428 & 0.3333 & 0.5161 & 0.7142 & 0.8346 & 0.9058 \\
                    4 &  &  & 0.0666 & 0.3548 & 0.6190 & 0.7795 & 0.8745 \\
                    5 &  &  & 0.0666 & 0.3548 & 0.5714 & 0.7244 & 0.8431 \\
                    6 &  &  & 0.0666 & 0.1935 & 0.4761 & 0.6692 & 0.8117 \\
                    7 &  &  & 0.0666 & 0.1935 & 0.3809 & 0.6141 & 0.7803 \\
                    8 &  &  &  & 0.0322 & 0.2857 & 0.5590 & 0.7490 \\
                    9 &  &  &  & 0.0322 & 0.2857 & 0.5590 & 0.7333 \\
                    10 &  &  &  & 0.0322 & 0.2857 & 0.5039 & 0.7019 \\
                    11 &  &  &  & 0.0322 & 0.2539 & 0.4488 & 0.6705 \\
                    12 &  &  &  & 0.0322 & 0.1587 & 0.3937 & 0.6392 \\
                    13 &  &  &  & 0.0322 & 0.1587 & 0.3937 & 0.6078 \\
                    14 &  &  &  & 0.0322 & 0.1111 & 0.3385 & 0.5764 \\
                    15 &  &  &  & 0.0322 & 0.1111 & 0.2834 & 0.5450 \\
                    16 &  &  &  &  & 0.0158 & 0.2283 & 0.5137 \\
                    17 &  &  &  &  & 0.0158 & 0.2283 & 0.5137 \\
                    18 &  &  &  &  & 0.0158 & 0.2283 & 0.5137 \\
                    19 &  &  &  &  & 0.0158 & 0.2283 & 0.4823 \\
                    20 &  &  &  &  & 0.0158 & 0.2283 & 0.4509 \\
                    21 &  &  &  &  & 0.0158 & 0.2283 & 0.4509 \\
                    22 &  &  &  &  & 0.0158 & 0.1732 & 0.4196 \\
                    23 &  &  &  &  & 0.0158 & 0.1732 & 0.3882 \\
                    24 &  &  &  &  & 0.0158 & 0.1181 & 0.3568 \\
                    25 &  &  &  &  & 0.0158 & 0.1181 & 0.3568 \\
                    26 &  &  &  &  & 0.0158 & 0.1181 & 0.3411 \\
                    27 &  &  &  &  & 0.0158 & 0.1181 & 0.3098 \\
                    28 &  &  &  &  & 0.0158 & 0.0629 & 0.2784 \\
                    29 &  &  &  &  & 0.0158 & 0.0629 & 0.2784 \\
                    30 &  &  &  &  & 0.0158 & 0.0629 & 0.2470 \\
                    31 &  &  &  &  & 0.0158 & 0.0629 & 0.2156 \\
                    32 &  &  &  &  &  & 0.0078 & 0.1843 \\
                    33 &  &  &  &  &  & 0.0078 & 0.1843 \\
                    34 &  &  &  &  &  & 0.0078 & 0.1843 \\
                    35 &  &  &  &  &  & 0.0078 & 0.1843 \\
                    36 &  &  &  &  &  & 0.0078 & 0.1843 \\
                    37 &  &  &  &  &  & 0.0078 & 0.1843 \\
                    38 &  &  &  &  &  & 0.0078 & 0.1843 \\
                    39 &  &  &  &  &  & 0.0078 & 0.1843 \\
                    40 &  &  &  &  &  & 0.0078 & 0.1843 \\
                    41 &  &  &  &  &  & 0.0078 & 0.1843 \\
                    42 &  &  &  &  &  & 0.0078 & 0.1843 \\
                    43 &  &  &  &  &  & 0.0078 & 0.1764 \\
                    44 &  &  &  &  &  & 0.0078 & 0.1450 \\
                    45 &  &  &  &  &  & 0.0078 & 0.1450 \\
                    46 &  &  &  &  &  & 0.0078 & 0.1137 \\
                    47 &  &  &  &  &  & 0.0078 & 0.1137 \\
                    48 &  &  &  &  &  & 0.0078 & 0.0823 \\
                    49 &  &  &  &  &  & 0.0078 & 0.0823 \\
                    50 &  &  &  &  &  & 0.0078 & 0.0823 \\
                    51 &  &  &  &  &  & 0.0078 & 0.0823 \\
                    52 &  &  &  &  &  & 0.0078 & 0.0823 \\
                    53 &  &  &  &  &  & 0.0078 & 0.0823 \\
                    54 &  &  &  &  &  & 0.0078 & 0.0823 \\
                    55 &  &  &  &  &  & 0.0078 & 0.0823 \\
                    56 &  &  &  &  &  & 0.0078 & 0.0509 \\
                    57 &  &  &  &  &  & 0.0078 & 0.0509 \\
                    58 &  &  &  &  &  & 0.0078 & 0.0509 \\
                    59 &  &  &  &  &  & 0.0078 & 0.0509 \\
                    60 &  &  &  &  &  & 0.0078 & 0.0352 \\
                    61 &  &  &  &  &  & 0.0078 & 0.0352 \\
                    62 &  &  &  &  &  & 0.0078 & 0.0352 \\
                    63 &  &  &  &  &  & 0.0078 & 0.0352 \\
                    64 &  &  &  &  &  &  & 0.0039 \\
                    65 &  &  &  &  &  &  & 0.0039 \\
                    66 &  &  &  &  &  &  & 0.0039 \\
                    67 &  &  &  &  &  &  & 0.0039 \\
                    68 &  &  &  &  &  &  & 0.0039 \\
                    69 &  &  &  &  &  &  & 0.0039 \\
                    70 &  &  &  &  &  &  & 0.0039 \\
                    71 &  &  &  &  &  &  & 0.0039 \\
                    72 &  &  &  &  &  &  & 0.0039 \\
                    73 &  &  &  &  &  &  & 0.0039 \\
                    74 &  &  &  &  &  &  & 0.0039 \\
                    75 &  &  &  &  &  &  & 0.0039 \\
                    76 &  &  &  &  &  &  & 0.0039 \\
                    77 &  &  &  &  &  &  & 0.0039 \\
                    78 &  &  &  &  &  &  & 0.0039 \\
                    79 &  &  &  &  &  &  & 0.0039 \\
                    80 &  &  &  &  &  &  & 0.0039 \\
                    81 &  &  &  &  &  &  & 0.0039 \\
                    82 &  &  &  &  &  &  & 0.0039 \\
                    83 &  &  &  &  &  &  & 0.0039 \\
                    84 &  &  &  &  &  &  & 0.0039 \\
                    85 &  &  &  &  &  &  & 0.0039 \\
                    86 &  &  &  &  &  &  & 0.0039 \\
                    87 &  &  &  &  &  &  & 0.0039 \\
                    88 &  &  &  &  &  &  & 0.0039 \\
                    89 &  &  &  &  &  &  & 0.0039 \\
                    90 &  &  &  &  &  &  & 0.0039 \\
                    91 &  &  &  &  &  &  & 0.0039 \\
                    92 &  &  &  &  &  &  & 0.0039 \\
                    93 &  &  &  &  &  &  & 0.0039 \\
                    94 &  &  &  &  &  &  & 0.0039 \\
                    95 &  &  &  &  &  &  & 0.0039 \\
                    96 &  &  &  &  &  &  & 0.0039 \\
                    97 &  &  &  &  &  &  & 0.0039 \\
                    98 &  &  &  &  &  &  & 0.0039 \\
                    99 &  &  &  &  &  &  & 0.0039 \\
                    100 &  &  &  &  &  &  & 0.0039 \\
                    101 &  &  &  &  &  &  & 0.0039 \\
                    102 &  &  &  &  &  &  & 0.0039 \\
                    103 &  &  &  &  &  &  & 0.0039 \\
                    104 &  &  &  &  &  &  & 0.0039 \\
                    105 &  &  &  &  &  &  & 0.0039 \\
                    106 &  &  &  &  &  &  & 0.0039 \\
                    107 &  &  &  &  &  &  & 0.0039 \\
                    108 &  &  &  &  &  &  & 0.0039 \\
                    109 &  &  &  &  &  &  & 0.0039 \\
                    110 &  &  &  &  &  &  & 0.0039 \\
                    111 &  &  &  &  &  &  & 0.0039 \\
                    112 &  &  &  &  &  &  & 0.0039 \\
                    113 &  &  &  &  &  &  & 0.0039 \\
                    114 &  &  &  &  &  &  & 0.0039 \\
                    115 &  &  &  &  &  &  & 0.0039 \\
                    116 &  &  &  &  &  &  & 0.0039 \\
                    117 &  &  &  &  &  &  & 0.0039 \\
                    118 &  &  &  &  &  &  & 0.0039 \\
                    119 &  &  &  &  &  &  & 0.0039 \\
                    120 &  &  &  &  &  &  & 0.0039 \\
                    121 &  &  &  &  &  &  & 0.0039 \\
                    122 &  &  &  &  &  &  & 0.0039 \\
                    123 &  &  &  &  &  &  & 0.0039 \\
                    124 &  &  &  &  &  &  & 0.0039 \\
                    125 &  &  &  &  &  &  & 0.0039 \\
                    126 &  &  &  &  &  &  & 0.0039 \\
                    127 &  &  &  &  &  &  & 0.0039 \\
                    \hline
                \end{longtable}
            \end{center}

            Попробуем ответить на вопрос о том, какое значение $t$ надо выбирать для заданного $n$. Введем следующие величины: $s$ - скорость передачи одного бита (биты в секунду). $A$ - кол-во бит, которое надо передать. \\

            Для простоты будем считать, что если проихошло больше чем $t$ ошибок, то мы получаем отказ от декодирования и запрашиваем это же сообщение заново. \\

            Разделим $A$ на порции по k бит:
            $$
                \frac{A}{k}
            $$
            Каждая такая порция после кодирования будет содержать $n$ бит, т.е. суммарное кол-во бит, которое нужно передать:
            $$
                \frac{A}{k}n = \left\{r=\frac{k}{n}\right\} =  \frac{A}{r}
            $$
            Время, которое мы потратим на передачу всей этой информации, если считать, что ошибок не произошло (обозначим это время за $T$):
            $$
                T \defeq \frac{A}{r}\cdot\frac{1}{s} = \frac{A}{rs}
            $$
            Далее рассмотрим случайную величину $\xi$ равную номеру первого успеха в серии испытаний Бернули (известно, что такая случайная величина имеет геометрическое распределение). Под испытанием будем понимать попытку передачи одного кодового слова длины $n$, под успехом испытания будем понимать факт передачи сообщения, с произошедшим количеством ошибок $\in [0, t]$. Если ошибок произошло больше чем $t$, то мы условились, что мы получаем отказ от декодирования и принимающая сторона запрашивает кодовое слово еще раз, это будем считать за неудачу. \\

            Понятно, что вероятность успеха в каждом испытании равна $\frac{t}{n}$. Т.е. случайная величина $\eta$ равная 0 или 1 (неудаче или успеху в соответствующем испытании):
            $$
                \eta = \left\{\begin{array}{ll}
                    1, & p = \frac{t}{n} \\
                    0, & q = 1 - p \\
                \end{array}\right.
            $$

            И мат. ожидание $\xi$, как случайной величины имеющей геометрическое распределение:
            $$
                \E(\xi) = \frac{1}{p} = \frac{n}{t}
            $$

            Таким образом кодовое слово будет отправлено лишь с попытки равной $\xi$. И если мы умножим $T$ на мат. ожидание величины $\xi$, то мы получим ожидаемое время передачи с учетом произошедших неуспехов при передаче:
            $$
                T\cdot\E(\xi) = \frac{An}{rst} \sim \frac{1}{rt} \defeq \lambda(t)
            $$
            Подытоживая, получаем, что $t$ надо выбирать таким образом, чтобы минимизировать функцию $\lambda$. \\

            Приведем графики $\lambda(t)$ для $ n = 15, 31 63, 127, 255$.

            \TimePlot{15}{xtick = data,}{*}{
                (1, 1.3636363636363638)
                (2, 1.0714285714285714)
                (3, 1.0)
                (4, 3.75)
                (5, 3.0)
                (6, 2.5)
                (7, 2.142857142857143)
            }

            \TimePlot{31}{xtick = data,}{*}{
                (1, 1.1923076923076923)
                (2, 0.7380952380952381)
                (3, 0.6458333333333334)
                (4, 0.7045454545454545)
                (5, 0.5636363636363636)
                (6, 0.8611111111111113)
                (7, 0.7380952380952381)
                (8, 3.875)
                (9, 3.444444444444445)
                (10, 3.1)
                (11, 2.818181818181818)
                (12, 2.5833333333333335)
                (13, 2.384615384615385)
                (14, 2.2142857142857144)
                (15, 2.0666666666666664)
            }

            \TimePlot{63}{}{*}{
                (1, 1.1052631578947367)
                (2, 0.6176470588235294)
                (3, 0.4666666666666667)
                (4, 0.40384615384615385)
                (5, 0.35000000000000003)
                (6, 0.35000000000000003)
                (7, 0.375)
                (8, 0.4375)
                (9, 0.38888888888888895)
                (10, 0.35000000000000003)
                (11, 0.35795454545454547)
                (12, 0.525)
                (13, 0.48461538461538467)
                (14, 0.6428571428571429)
                (15, 0.6000000000000001)
                (16, 3.9375)
                (17, 3.7058823529411766)
                (18, 3.5)
                (19, 3.3157894736842106)
                (20, 3.1500000000000004)
                (21, 3.0)
                (22, 2.8636363636363638)
                (23, 2.739130434782609)
                (24, 2.625)
                (25, 2.52)
                (26, 2.4230769230769234)
                (27, 2.3333333333333335)
                (28, 2.25)
                (29, 2.1724137931034484)
                (30, 2.1)
                (31, 2.032258064516129)
            }

            \TimePlot{127}{}{}{
                (1, 1.0583333333333333)
                (2, 0.5619469026548672)
                (3, 0.39937106918238996)
                (4, 0.3207070707070707)
                (5, 0.2760869565217391)
                (6, 0.24901960784313726)
                (7, 0.23260073260073258)
                (8, 0.22359154929577466)
                (9, 0.19874804381846636)
                (10, 0.1984375)
                (11, 0.20255183413078148)
                (12, 0.21166666666666664)
                (13, 0.19538461538461535)
                (14, 0.21096345514950163)
                (15, 0.23518518518518522)
                (16, 0.27370689655172414)
                (17, 0.25760649087221094)
                (18, 0.24329501915708812)
                (19, 0.2304900181488203)
                (20, 0.2189655172413793)
                (21, 0.20853858784893267)
                (22, 0.26239669421487605)
                (23, 0.2509881422924901)
                (24, 0.3527777777777778)
                (25, 0.33866666666666667)
                (26, 0.32564102564102565)
                (27, 0.3135802469135802)
                (28, 0.5669642857142857)
                (29, 0.5474137931034483)
                (30, 0.5291666666666667)
                (31, 0.5120967741935484)
                (32, 3.96875)
                (33, 3.8484848484848486)
                (34, 3.7352941176470593)
                (35, 3.6285714285714286)
                (36, 3.5277777777777777)
                (37, 3.4324324324324325)
                (38, 3.3421052631578947)
                (39, 3.2564102564102564)
                (40, 3.175)
                (41, 3.097560975609756)
                (42, 3.023809523809524)
                (43, 2.953488372093023)
                (44, 2.8863636363636362)
                (45, 2.8222222222222224)
                (46, 2.760869565217391)
                (47, 2.702127659574468)
                (48, 2.6458333333333335)
                (49, 2.5918367346938775)
                (50, 2.54)
                (51, 2.4901960784313726)
                (52, 2.4423076923076925)
                (53, 2.3962264150943398)
                (54, 2.3518518518518516)
                (55, 2.309090909090909)
                (56, 2.267857142857143)
                (57, 2.2280701754385968)
                (58, 2.189655172413793)
                (59, 2.152542372881356)
                (60, 2.1166666666666667)
                (61, 2.081967213114754)
                (62, 2.0483870967741935)
                (63, 2.015873015873016)
            }

            \TimePlot{255}{}{}{
                (1, 1.0323886639676114)
                (2, 0.5334728033472803)
                (3, 0.36796536796536794)
                (4, 0.2858744394618834)
                (5, 0.23720930232558138)
                (6, 0.20531400966183572)
                (7, 0.18305814788226848)
                (8, 0.1668848167539267)
                (9, 0.15151515151515152)
                (10, 0.1424581005586592)
                (11, 0.13556618819776714)
                (12, 0.1303680981595092)
                (13, 0.12655086848635236)
                (14, 0.1239067055393586)
                (15, 0.1223021582733813)
                (16, 0.12166030534351147)
                (17, 0.11450381679389314)
                (18, 0.10814249363867685)
                (19, 0.10911424903722722)
                (20, 0.1108695652173913)
                (21, 0.10559006211180125)
                (22, 0.10832625318606626)
                (23, 0.11198945981554677)
                (24, 0.11675824175824176)
                (25, 0.11208791208791208)
                (26, 0.11273209549071618)
                (27, 0.11954992967651194)
                (28, 0.1282696177062374)
                (29, 0.1238465274405051)
                (30, 0.1349206349206349)
                (31, 0.1495601173020528)
                (32, 0.1695478723404255)
                (33, 0.16441005802707928)
                (34, 0.15957446808510636)
                (35, 0.15501519756838905)
                (36, 0.15070921985815602)
                (37, 0.14663599769982746)
                (38, 0.1427771556550952)
                (39, 0.13911620294599017)
                (40, 0.1356382978723404)
                (41, 0.13233004670472234)
                (42, 0.12917933130699086)
                (43, 0.1317829457364341)
                (44, 0.1566339066339066)
                (45, 0.15315315315315314)
                (46, 0.1911544227886057)
                (47, 0.18708730741012475)
                (48, 0.25297619047619047)
                (49, 0.2478134110787172)
                (50, 0.24285714285714288)
                (51, 0.23809523809523808)
                (52, 0.23351648351648352)
                (53, 0.22911051212938008)
                (54, 0.22486772486772486)
                (55, 0.2207792207792208)
                (56, 0.35027472527472525)
                (57, 0.3441295546558705)
                (58, 0.3381962864721485)
                (59, 0.3324641460234681)
                (60, 0.4722222222222222)
                (61, 0.4644808743169399)
                (62, 0.456989247311828)
                (63, 0.4497354497354497)
                (64, 3.984375)
                (65, 3.9230769230769234)
                (66, 3.863636363636364)
                (67, 3.805970149253731)
                (68, 3.75)
                (69, 3.695652173913044)
                (70, 3.6428571428571423)
                (71, 3.5915492957746475)
                (72, 3.5416666666666665)
                (73, 3.493150684931507)
                (74, 3.4459459459459465)
                (75, 3.4)
                (76, 3.3552631578947367)
                (77, 3.311688311688312)
                (78, 3.269230769230769)
                (79, 3.2278481012658227)
                (80, 3.1875)
                (81, 3.1481481481481484)
                (82, 3.1097560975609757)
                (83, 3.072289156626506)
                (84, 3.0357142857142856)
                (85, 3.0)
                (86, 2.9651162790697674)
                (87, 2.9310344827586206)
                (88, 2.897727272727273)
                (89, 2.865168539325843)
                (90, 2.8333333333333335)
                (91, 2.802197802197802)
                (92, 2.7717391304347827)
                (93, 2.741935483870968)
                (94, 2.712765957446808)
                (95, 2.6842105263157894)
                (96, 2.65625)
                (97, 2.6288659793814433)
                (98, 2.602040816326531)
                (99, 2.5757575757575757)
                (100, 2.55)
                (101, 2.5247524752475248)
                (102, 2.5)
                (103, 2.4757281553398056)
                (104, 2.451923076923077)
                (105, 2.428571428571429)
                (106, 2.405660377358491)
                (107, 2.383177570093458)
                (108, 2.361111111111111)
                (109, 2.3394495412844036)
                (110, 2.318181818181818)
                (111, 2.2972972972972974)
                (112, 2.2767857142857144)
                (113, 2.256637168141593)
                (114, 2.236842105263158)
                (115, 2.217391304347826)
                (116, 2.1982758620689657)
                (117, 2.1794871794871797)
                (118, 2.1610169491525424)
                (119, 2.142857142857143)
                (120, 2.125)
                (121, 2.107438016528926)
                (122, 2.0901639344262297)
                (123, 2.073170731707317)
                (124, 2.056451612903226)
                (125, 2.04)
                (126, 2.0238095238095237)
                (127, 2.0078740157480315)
            }

            Таким образом, для $n = 127$ оптимальным значением оказалось $t = 13$, а для $n = 255$, $t = 21$. Но как видно по графикам нет особо сильной принципиальности в выборе именно таких $t$. Так для $n = 127$ можно позволить себе выбирать $t$ в промежутке от 9 до 23 и если мы имеем дело с каналом связи с большим количеством помех, то можно отдать предпочтение большему $t$. \\

            Можно заметить, что на графике зависимости $r$ от $t$ есть участки, когда $r$ остается постоянным некоторое время. Для таких участков очевидно, что если мы и выбрали код, исправляющий кол-во ошибок для $t$ из этого участка, то надо выбирать наибольшее $t$ для этого участка постоянства $r$. Потому что таким образом характеристика кода не ухудшается, а теоретическое кол-во ошибок, которое может исправлять код растет.

        % end \subsection{Скорость БЧХ кода и выбор оптимального кода для заданного n}

        \subsection{БЧХ коды, для которых истинное расстояние больше чем 2t+1}
            Приведем некотрые примеры БЧХ кодов для которых путем перебора было получено, что их истинное расстояние больше, чем $2t + 1$:
            \begin{center}
                \begin{tabular}{|c|c|c|c|}
                    % \hline
                    % \multicolumn{4}{|c|}{Время в секундах} \\
                    \hline
                    $n$ & $t$ & $2t + 1$ & истинное расстояние кода \\
                    \hline
                    7 & 2 & 5 & 7 \\
                    \hline
                    15 & 4 & 9 & 15 \\
                    \hline
                    15 & 5 & 11 & 15 \\
                    \hline
                    15 & 6 & 13 & 15 \\
                    \hline
                    31 & 4 & 9 & 11 \\
                    \hline
                    31 & 6 & 13 & 15 \\
                    \hline
                    31 & 8 & 17 & 31 \\
                    \hline
                    31 & 9 & 19 & 31 \\
                    \hline
                    31 & 10 & 21 & 31 \\
                    \hline
                    31 & 11 & 23 & 31 \\
                    \hline
                    31 & 12 & 25 & 31 \\
                    \hline
                    31 & 13 & 27 & 31 \\
                    \hline
                    31 & 14 & 28 & 31 \\
                    \hline
                \end{tabular}
            \end{center}
            Если посмотреть на графики зависимости $r$ от $t$, то можно увидеть, что есть моменты, когда мы увеличиваем $t$, а скорость БЧХ-кода $r$ при этом не меняется. Логически это воспринимается, как то, что мы требуем от кода большего числа исправляемых ошибок, а код при этом не ухудшает свои характеристики. И интуитивно понятно, что если и существуют коды, с истинным кодовым расстоянием больше чем $2t + 1$, то это такие коды, для которых $t$ это не крайняя правая точка из промежутка постоянства $r$. Что подтвердилось экспериментальным путем, т.е. табличкой, приведенной выше.
        % end \subsection{БЧХ коды для которых истинное расстояние больше чем 2t+1}

        \subsection{Сравнение времени работы декодирования БЧХ-кода с помощью PGZ и на основе расширенного алгоритма Евклида}

        % end \subsection{Сравнение времени работы декодирования БЧХ-кода с помощью PGZ и на основе расширенного алгоритма Евклида}


    % end \section{Результаты выполнения}


            
        
    \section{Постановка задачи}
        \textbf{Задача:} Разработать параллельную программу с использованием технологии MPI, реализующую
        алгоритм умножения плотных матриц на C=AB. Тип данных – double. Провести
        исследование эффективности разработанной программы на системе Blue Gene/P. \\

        \textbf{Формат командной строки:} Параметры, передаваемые в командной строке:
        \begin{itemize}
            \setlength{\itemsep}{1pt}
            \setlength{\parskip}{0pt}
            \setlength{\parsep}{0pt}
            \item имя файла - матрица A размером n x n
            \item имя файла - матрица B размером n x n
            \item имя файла - результат, матрица C
        \end{itemize}
        Формат задания матрицы A – как в первом задании. \\

        \textbf{Требуется:} \\
        1. Разработать параллельную программу с использованием технологии MPI.
        Предусмотреть равномерное распределение элементов матриц блоками. Для
        организации работы с файлами использовать функции MPI для работы с
        параллельным вводом-выводом. \\
        2. Исследовать эффективность разработанной программы в зависимости от размеров
        матрицы и количества используемых процессов. Построить графики времени работы,
        ускорения и эффективности разработанной программы. Время на ввод/вывод данных
        не включать. \\
        3. Исследовать эффективность использования параллельной работы с файлами. Для
        каждого из вариантов построить графики накладных расходов, связанных с
        вводом/выводом. \\
        4. Исследовать влияние мэппинга параллельной программы на время работы
        программы. \\
        5. Построить таблицы: времени, ускорения, эффективности.

    \section{Результаты выполнения}
        Мэппинг производился на 125 процессах.
        \begin{center}
            \begin{tabular}{|c|c|c|c|c|c|c|}
                \hline
                \multicolumn{7}{|c|}{Время в секундах} \\
                \hline
                M & N & Мэппинг & 1 & 8 & 64 & 125 \\
                \hline
                1024 & 1024 & 0.207990 & 23.581691 & 2.895441 & 0.366487 & 0.207839 \\
                \hline
                2048 & 2048 & 1.559735 & 199.969342 & 23.693282 & 2.897212 & 1.559119 \\
                \hline
                4096 & 4096 & 12.456126 & 1621.846709 & 200.413686 & 23.699439 & 12.456071 \\
                \hline
            \end{tabular}
        \end{center}

        \begin{center}
            \begin{tabular}{|c|c|c|c|c|c|c|}
                \hline
                \multicolumn{7}{|c|}{Ускорение} \\
                \hline
                M & N & Мэппинг & 1 & 8 & 64 & 125 \\
                \hline
                1024 & 1024 & 113.378965335 & 1.0 & 8.14442117798 & 64.3452318909 & 113.461337862 \\
                \hline
                2048 & 2048 & 128.207254437 & 1.0 & 8.43991735716 & 69.0213011682 & 128.257908473 \\
                \hline
                4096 & 4096 & 130.204744958 & 1.0 & 8.09249478601 & 68.4339704834 & 130.20531988 \\
                \hline
            \end{tabular}
        \end{center}

        \begin{center}
            \begin{tabular}{|c|c|c|c|c|c|c|}
                \hline
                \multicolumn{7}{|c|}{Эффективность} \\
                \hline
                M & N & Мэппинг & 1 & 8 & 64 & 125 \\
                \hline
                1024 & 1024 & 0.907031722679 & 1.0 & 1.01805264725 & 1.0053942483 & 0.9076907029 \\
                \hline
                2048 & 2048 & 1.0256580355 & 1.0 & 1.05498966965 & 1.07845783075 & 1.02606326778 \\
                \hline
                4096 & 4096 & 1.04163795967 & 1.0 & 1.01156184825 & 1.0692807888 & 1.04164255904 \\
                \hline
            \end{tabular}
        \end{center}

        \begin{center}
            \begin{tabular}{|c|c|c|c|c|c|c|}
                \hline
                \multicolumn{7}{|c|}{Время на ввод/вывод} \\
                \hline
                M & N & Мэппинг & 1 & 8 & 64 & 125 \\
                \hline
                1024 & 1024 & 0.363743 & 0.482833 & 0.209512 & 0.611490 & 0.596685 \\
                \hline
                2048 & 2048 & 1.139824 & 0.850338 & 1.770940 & 1.073736 & 1.392719 \\
                \hline
                4096 & 4096 & 5.056136 & 2.318312 & 3.245491 & 5.167992 & 5.520358 \\
                \hline
            \end{tabular}
        \end{center}

    \section{Основные выводы}
        Исследования показывают, что при большем количестве процессов скорость работы
        программы повышается.
    
    

\end{document}